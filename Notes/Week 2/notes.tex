\documentclass[12pt]{article}

\usepackage{amsmath}
\usepackage{graphicx}
\usepackage{tabularx}
\usepackage{multicol}
\usepackage{algpseudocode}
\usepackage{algorithm}

% Geometry 
\usepackage{geometry}
\geometry{letterpaper, left=15mm, top=20mm, right=15mm, bottom=20mm}

% Fancy Header
\usepackage{fancyhdr}
\renewcommand{\footrulewidth}{0.4pt}
\pagestyle{fancy}
\fancyhf{}
\chead{CSC 360 - Analysis of Algorithms}
\lfoot{CALU Fall 2021}
\rfoot{RDK}

% Add vertical spacing to tables
\renewcommand{\arraystretch}{1.4}

% Macros
\newcommand{\definition}[1]{\underline{\textbf{#1}}}

\newenvironment{rcases}
  {\left.\begin{aligned}}
  {\end{aligned}\right\rbrace}

% Begin Document
\begin{document}

\begin{itemize}

    \item Read Chapters 7 and 8 for Week 3. Quiz in Week 3.

    \item What governs how fast an algorithm can run?
    \begin{itemize}
        \item The complexity of the problem; a problem has a theoretical best solution
        \item The efficiency of the algorithm; an algorithm can give a close enough solution in a fraction of the time
    \end{itemize}

    \item Sometimes writing a program that always guesses the correct solution is not always good enough.
    \begin{itemize}
        \item Time complexity; a program taking too long for its answer; keep in mind timescales
    \end{itemize}

\end{itemize}

\section*{Properites of Exponents}

\begin{itemize}

    \item $x^a * x^b = x^{a+b}$
    \item $(x^a)^b = x^{ab}$
    \item $x^a / x^b = x^{a-b}$

\end{itemize}

\section*{Properties of Logarithms}
\begin{itemize}

    \item $x^a = b$ iff $log_x b = a$

    \item $log_a b = log_c b / log_c a$ if $a, b, c > 0, a \neq 0$
    \begin{itemize}
        \item Let $x = log_c b$, $y = log_c a$, $z = log_a b$
        \item $b = c^x, b=a^z \rightarrow b = c^x = a^z$
        \item $a = c^y$
        \item $b = c^x = c^{yz}$
        \item $x = yz \rightarrow z = x/y$
    \end{itemize}

    \item $logab = loga + logb$
    \begin{itemize}
        \item Let $x = log_2 a, y = log_2 b, z = log_2 ab$ if $a, b > 0$
        \item $2^x = a, 2^y = b, 2^z = ab$
        \item $2^x * 2^y = 2^z$
        \item $2^{x+y} = 2^z$
        \item $x + y = z$
    \end{itemize}

    \item $log_2 (a/b) = log_2 a - log_2 b$ if $a, b > 0$
    \begin{itemize}
        \item Let $x = log_2 a, y = log_2 b, z = log_2 (a/b)$
        \item $2^x = a, 2^y = b, 2^z = a / b$
        \item $2^x / 2^y = 2^z$
        \item $x-y = z$
    \end{itemize}

\end{itemize}

\pagebreak

\section*{Disproof}
\begin{itemize}
    \item Only need to find one counter example
    \item Example: $n! < n^2 + n^5$ \\
    \begin{tabular}{l l l}
        Pick $n=9$ & & \\
        $9! = 362880$ & $9^2 = 81$ & $9^5 = 59130$ \\
        $362880 < 81 + 59130$ & FALSE! &
    \end{tabular}
\end{itemize}

\section*{Proof by Induction}
\begin{itemize}
    \item Use proof by Induction for discrete types only.
    \item Happens in 3 steps:
    \begin{enumerate}
        \item Start with a (very small) base case.
        \item Inductive Hypothesis: ``Since we showed a base case in Step 1, let's assume it holds true for values through $k$''
        \item Show that if the Hypothesis works for value $k$, the Hypothesis holds for $k+1$
    \end{enumerate}
\end{itemize}

\subsection*{Example of Induction}

\[ \sum_{i=1}^{n} i = \frac{n(n+1)}{2} \]

\begin{enumerate}
    \item Let $n=1 \rightarrow 1 = 1$
    \item Assume the property is true for $1, 2, \ldots, k$
    \item Show if true for $k$, then true for $k+1$
    \\
    \begin{tabular}{l l l}
        $\sum_{i=1}^{k+1} i$ & $=$ & $\sum_{i=1}^{k} + k + 1$ \\
        & $=$ & $\frac{k(k+1)}{2} + k + 1$ \\
        & $=$ & $\frac{k(k+1)}{2} + \frac{2(k+1)}{2}$ \\
        & $=$ & $\frac{k(k+1) + 2(k+1)}{2}$ \\
        & $=$ & $\frac{(k+1)(k+2)}{2}$ \\
        & $=$ & $\frac{(k+1)(k+1+1)}{2}$ \\
    \end{tabular}


\end{enumerate}

\pagebreak

\subsection*{Example 2 of Induction}

\[ \sum_{i=1}^{n} i(i+1) = \frac{n(n+1)(n+2)}{3} \]

\begin{enumerate}
    \item Let $n=1 \rightarrow 2 = 2$
    \item Assume the property is true for $1, 2, \ldots, k$
    \item Show if true for $k$, then true for $k+1$
    \\
    \begin{tabular}{l l l}
        $\sum_{i=1}^{k+1} i(i+1)$ & $=$ & $\sum_{i=1}^{k} i(i+1) + (k+1)(k+2)$ \\
        & $=$ & $\frac{k(k+1)(k+2)}{3} + (k+1)(k+2)$ \\
        & $=$ & $\frac{k(k+1)(k+2)}{3} + \frac{3(k+1)(k+2)}{3}$ \\
        & $=$ & $\frac{k(k+1)(k+2) + 3(k+1)(k+2)}{3}$ \\
        & $=$ & $\frac{(k+1)(k+2)(k+3)}{3}$ \\
        & $=$ & $\frac{(k+1)(k+1+1)(k+1+2)}{3}$ \\
    \end{tabular}

\end{enumerate}

\subsection*{Example 3 of Induction}

Fibonnaci Numbers $\rightarrow F_0 = 1, F_1 = 1, \ldots, (F_i = F_{i-1} + F_{i-2})$

Prove that $F_i < ({\frac{5}{3}})^i$, for $i \geq 1$

\begin{enumerate}

    \item $F_1 = 1 < (\frac{5}{3})^1$
    \item Assume this is true for $1, \ldots, k$
    \item Show if true for $k$, then true for $k+1$
    \\
    \begin{tabular}{l l l l l}
        $F_{k+1}$ & $=$ & $F_k + F_{k-1}$ && \\
        $F_{k+1}$ & $<$ & $(\frac{5}{3})^k + (\frac{5}{3})^{k-1}$ && \\
        $F_{k+1}$ & $<$ & $(\frac{3}{5})(\frac{5}{3})^{k+1} + (\frac{3}{5})^2(\frac{5}{3})^{k+1}$ && \\
        $F_{k+1}$ & $<$ & $(\frac{5}{3})^{k+1} * ((\frac{3}{5})^2+(\frac{3}{5}))$ && \\
        $F_{k+1}$ & $<$ & $(\frac{5}{3})^{k+1} * (\frac{15}{25} + \frac{9}{25})$ && \\
        $F_{k+1}$ & $<$ & $(\frac{5}{3})^{k+1} * \frac{24}{25}$ & $<$ & $({\frac{5}{3}})^{k+1}$ \\
    \end{tabular}

\end{enumerate}

\end{document}