\documentclass[12pt]{article}

\usepackage{tikz}
\usepackage{amsmath}
\usepackage{graphicx}
\usepackage{tabularx}
\usepackage{multicol}
\usepackage{algpseudocode}
\usepackage{algorithm}

% Geometry 
\usepackage{geometry}
\geometry{letterpaper, left=15mm, top=20mm, right=15mm, bottom=20mm}

% Fancy Header
\usepackage{fancyhdr}
\renewcommand{\footrulewidth}{0.4pt}
\pagestyle{fancy}
\fancyhf{}
\chead{CSC 360 - Analysis of Algorithms}
\lfoot{CALU Fall 2021}
\rfoot{RDK}

% Add vertical spacing to tables
\renewcommand{\arraystretch}{1.4}

% Macros
\newcommand{\definition}[1]{\underline{\textbf{#1}}}

\newenvironment{rcases}
  {\left.\begin{aligned}}
  {\end{aligned}\right\rbrace}

% Begin Document
\begin{document}

\section*{Notes Week 9}

\begin{itemize}

    \item Exam Week 10 - B-M 6:45pm, N-Y 6:00pm

\end{itemize}

\section*{Assignment - Minimum Dominating Set of a Graph}

\begin{itemize}

	\item Big ole Assignment
	\begin{itemize}
		\item Exploring $2^n$ runtime complexity
		\item Writing a paper that:
		\begin{enumerate}
			\item The problem
			\item The data
			\item The algorithm
		\end{enumerate}
	\end{itemize}

\end{itemize}

\section*{Minimum Domating Set of a Graph}

\begin{itemize}

	\item The nodes of a dominating set may not be unique
	\item The size of the set of nodes \textit{will} be unique

	\item \definition{Dominating Set of a Graph}: a subset of the vertices such that all nodes in the graph are either in the dominating set or have a neighbor that is in the dominating set.

	\item The best \textit{known} solution to this problem is to test every posssible combination of verticecs, aka \textbf{Brute Force}

	\item Use an array that is one larger than the number of nodes

	\item Using binaries numbers, can represent whether a node appears in the set using $0$s or $1$s.

	\item For each number 0 to $2^n$, pass the number to a function to determine if that number represents a dominating set

	\item \definition{Approximation Algorithm}: an algorithm we design to run quickly trying to get an answer that's close to the correct answer

\end{itemize}

\section*{Complexity Classes}

\begin{itemize}

	\item \definition{P}: the set of all Problems that can be solved in polynomial times - constant exponents

	\item \definition{NP}: the set of all problems whose solutions can be \textit{verified} in polynomial time

	\item $\textbf{P} \in \textbf{NP}$

	\item \definition{NP-Complete}: The hardest problems in \textbf{NP}.

	\item \definition{SAT} - The Satisfiability Problem. The original NP-Complete problem. 
	
	\item For \textbf{NP-Complete}: Not solved, but strongly believed to be, that these problems are not solvable in polynomial time.

	\item 

\end{itemize}

\end{document}