\documentclass[12pt]{article}

\usepackage{tikz}
\usepackage{amsmath}
\usepackage{graphicx}
\usepackage{tabularx}
\usepackage{multicol}
\usepackage{algpseudocode}
\usepackage{algorithm}

% Geometry 
\usepackage{geometry}
\geometry{letterpaper, left=15mm, top=20mm, right=15mm, bottom=20mm}

% Fancy Header
\usepackage{fancyhdr}
\renewcommand{\footrulewidth}{0.4pt}
\pagestyle{fancy}
\fancyhf{}
\chead{CSC 360 - Analysis of Algorithms}
\lfoot{CALU Fall 2021}
\rfoot{RDK}

% Add vertical spacing to tables
\renewcommand{\arraystretch}{1.4}

% Macros
\newcommand{\definition}[1]{\underline{\textbf{#1}}}

\newenvironment{rcases}
  {\left.\begin{aligned}}
  {\end{aligned}\right\rbrace}

% Begin Document
\begin{document}

\section*{Notes Week 12}

\subsection*{Minimum Spanning Trees}

\begin{itemize}

    \item spanning tree - $(V, T \subseteq E)$ - a connected, acylic subgraph containing all the vertices
    \begin{itemize}
      \item There are $V - 1$ edges (Weight of ST is the sum of all edge weights in T)
      \item A minimum spanning tree exists if and only if G is connected
      \item Greed is sometimes a good thing!
    \end{itemize}

    \item We will explore two greedy algorithms for finding a minimum spanning tree
    \begin{itemize}
      \item Kruskal (forest) - don't necessarily have a connected subtree until the end
      \begin{itemize}
        \item Sort edges first
        \item Start adding edges (don't create a cycle)
        \item Repeat step b until you have a spanning tree
      \end{itemize}

      \item Prim's (tree) - always have a connected subtree (we will see, this one is best)
      \begin{itemize}
        \item Start with any root
        \item Choose smallest weight edge coming out of it
        \item Choose smallest weight edge coming out of any connected vertex (don't create a cycle)
        \item Repeat step c until you have a spanning tree
      \end{itemize}
    \end{itemize}

\end{itemize}

\pagebreak

% TODO: Insert the graphs here

\pagebreak

\subsection*{Kruskal's Algorithm}

\begin{algorithm}

  \caption{Kruskal's Algorithm}

  \begin{algorithmic}

    \Procedure{Kruskal}{}

    \State $A \gets \emptyset$
    
    \For{each vertex $v \in G$}
      \State{$make_set(v)$}
    \EndFor

    \State{Sort edges of G in increasing order of weight}

    \For{each edge $(u,v)$}
      \If{$Find_Set(u) \neq Find_Set(v)$}
        \State{$A \gets A \cup \{(u,v)\}$}
        \State{$union(u,v)$}
      \EndIf
    \EndFor

    \EndProcedure

  \end{algorithmic}

\end{algorithm}

\begin{itemize}
  \item The sort takes the longest, so the running time is $O(E log(E))$
  \item Note that $E = O(V^2)$ - A compelte graph has $\frac{n(n-1)}{2}$ edges
  \item If two vertices are in the same set, and are connected, adding an edge between them would create a cycle
  \item A union operation joins two sets into one (adding an edge merges two trees into one)
  \item This is a greedy algorithm
  \item Find Set(u) runs in \textbf{Constant Time} - maintain, for each vertex, a pointer to the set it belongs to
  \item Union(u,v) runs in \textbf{Constant Time} - implement as linked lists, attach one to the end of another
\end{itemize}

\pagebreak

\subsection*{Prim's Algorithm}

% Insert the Algorithm here

\begin{itemize}
  \item First for-loop takes $O(V)$
  \item The while-loop runs $V$ times
  \item The dequeue statement takes $O(log(V))$, yielding $O(V log(V))$ since it is inside of the while-loop
  \item The for-loop inside of the while-loop can be simplified by noticing each edge can be processed once, with the heapify taking $O(log(V))$, yielding $O(E log(V))$
  \item Since $O(E log(V))$ dominates $O(V log(V))$, the final analysis is $O(E log(V))$
  \item This is also a greedy algorithm
  \item This is slightly better than Kruskal's
\end{itemize}





\end{document}