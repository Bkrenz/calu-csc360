\documentclass[12pt]{article}

\usepackage{amsmath}
\usepackage{graphicx}
\usepackage{tabularx}
\usepackage{multicol}
\usepackage{algpseudocode}
\usepackage{algorithm}

% Geometry 
\usepackage{geometry}
\geometry{letterpaper, left=15mm, top=20mm, right=15mm, bottom=20mm}

% Fancy Header
\usepackage{fancyhdr}
\renewcommand{\footrulewidth}{0.4pt}
\pagestyle{fancy}
\fancyhf{}
\chead{CSC 360 - Analysis of Algorithms}
\lfoot{CALU Fall 2021}
\rfoot{RDK}

% Add vertical spacing to tables
\renewcommand{\arraystretch}{1.4}

% Macros
\newcommand{\definition}[1]{\underline{\textbf{#1}}}

\newenvironment{rcases}
  {\left.\begin{aligned}}
  {\end{aligned}\right\rbrace}

% Begin Document
\begin{document}

Introduction to Algorithms - 3rd Edition - Thomas Cormen et.al.

\section*{Chapter 2}

\begin{itemize}

  \item The \definition{loop invariant} is the statement defining the properties of a loop at each iteration. It must satisfy three things:
  \begin{description}
      \item[Initialization] It is true prior to the first iteration of the loop.
      \item[Maintenance] If it is true before an iteration of the loop, it remains true before the next iteration.
      \item[Termination] When the loop terminates, the invariant gives us a useful property that helps show that the algorithm is correct.
  \end{description}

\end{itemize}

\end{document}