\documentclass[12pt]{article}

\usepackage{amsmath}
\usepackage{graphicx}
\usepackage{tabularx}
\usepackage{multicol}
\usepackage{algpseudocode}
\usepackage{algorithm}

% Geometry 
\usepackage{geometry}
\geometry{letterpaper, left=15mm, top=20mm, right=15mm, bottom=20mm}

% Fancy Header
\usepackage{fancyhdr}
\renewcommand{\footrulewidth}{0.4pt}
\pagestyle{fancy}
\fancyhf{}
\chead{CSC 360 - Analysis of Algorithms}
\lfoot{CALU Fall 2021}
\rfoot{RDK}

% Add vertical spacing to tables
\renewcommand{\arraystretch}{1.4}

% Macros
\newcommand{\definition}[1]{\underline{\textbf{#1}}}

\newenvironment{rcases}
  {\left.\begin{aligned}}
  {\end{aligned}\right\rbrace}

% Begin Document
\begin{document}


\section*{Notes Week 6}

\begin{itemize}

    \item Read Chapters 22, Chapter 9 for Week 7

\end{itemize}


\subsection*{Rules for Big O Notation (Upper Bounds)}

\begin{itemize}

    \item \textbf{Transitivity}: If $f = O(g)$ and $g = O(h)$ then $f = O(h)$

    \item \textbf{Sums}: If $f_1 = O(g_1)$ and $f_2 = O(g_2)$ then $f_1(n) + f_2(n) = O( max(f, g) )$
    
    \item \textbf{Products}: If $f_1 = O(g_1)$ and $f_2 = O(g_2)$ then $f_1 \times f_2 = O(g_1 \times g_2)$

    \item If the largest term is a polynomial of degree $k$, then the whole thing is $\Theta(n^k)$

    \item $log^kn = O(n) \rightarrow$ a log raised to any constant power is still just $O(n)$

    \item $\lim{n\to\infty} \frac{f(n)}{g(n)} \rightarrow$ If the limit is:
    \begin{itemize}
        \item \textbf{0} $\rightarrow f=o(g) \rightarrow g(n) > f(n) \rightarrow g$ dominates $f$
        \item \textbf{$c \neq 0$} $\rightarrow f = \Theta(g(n)) \rightarrow g(n) = \Theta(f(n))$
        \item \textbf{$\infty$} $\rightarrow g(n) = o(f(n))$
    \end{itemize}

    \item Example: $f(n) = \frac{n}{log(n), g(n) = n^{\frac{1}{2}}log^2n}$

\end{itemize}


\subsection*{Rules for Big $\Omega$ Notation (Lower Bounding)}

\begin{itemize}

    \item \textbf{Comparison-based Solution} $\omega(nlogn) \rightarrow$ it's been proven that comparison-based sorting cannot be faster
    
    \item Some problems are $\Omega(2^n) \rightarrow$ Super slow

\end{itemize}


\subsection*{Quick Maths}

\begin{itemize}

    \item $b^{log_ba} = a$

    \item $a^{log_bn} = n^{log_ba}$

    \item Any exponential function dominates any polynomial
    \begin{itemize}
        \item Exponential function $\rightarrow 2^n \rightarrow$ variable in the exponent
        \item Polynomial function $\rightarrow n^{234} \rightarrow$ exponent is a constant
    \end{itemize}

    \item $\sum(ca_k + b_k) = c\sum(a_k) + \sum(b_k)$

\end{itemize}


\section*{Recurrences - Recurrence Relations}

\begin{itemize}

    \item A \definition{Recurrence} is used when dealing with recursion. 
    
    \item Merge Sort: \\
    
    \begin{tabular}{l c c}
        \textbf{Case} & \textbf{Formula} & \textbf{Big O} \\
        base case & $T(n) = \Theta(1)$ & $\Theta(1)$ \\
        not base case & $2T(\frac{n}{2}) + \Theta(n)$ & ? 
    \end{tabular}

    \item How does $2T(\frac{n}{2})$ explain the recurrence of merge sort?
    \begin{itemize}
        \item Each division splits the array in half
        \item There are two pieces in each call, one for each half
        \item So the array is split in half, and each is operated on recursively
    \end{itemize}

    \item So what is the $O($merge-sort$)$ 
    \begin{itemize}

        \item $T(n) = 2T(\frac{n}{2}) + n \rightarrow$ Merge Sort Recurrence
        \item Solve using iteration method: ``Unfold the recurrence''
        \begin{enumerate}

            \item If there's a constant, write out that constant with square brackets empty to indicate something needs done
            \begin{equation}
                2[\ \ \ ] + n
            \end{equation}

            \item Plug the $\frac{n}{2}$ back into the original
            \begin{equation}
                2[ 2T(\frac{n}{4}) + \frac{n}{2} ] + n
            \end{equation}

            \item Multiply it out
            \begin{equation*}
                4T(\frac{n}{4}) + 2n
            \end{equation*}

            \item Take the previous iteration (2) and repeat
            \begin{equation}
                4[ 2T(\frac{n}{8}) + \frac{n}{4} ] + 2n
            \end{equation}

            \item Simplify
            \begin{equation*}
                8T(\frac{n}{8}) + 3n
            \end{equation*}

            \item Repeat the process until you can spot the general case. The next iteration would be
            \begin{equation}
                16T(\frac{n}{16}) + 4n
            \end{equation}

            \item Write down the general case in terms of $k$
            \begin{equation*}
                2^k(\frac{n}{2^k}) + kn
            \end{equation*}

            \item How many times does the recurrence occur? $k = log_2n$

            \item Plug in k
            \begin{equation*}
                2^{log_2n} T(\frac{n}{2^{log_2n}}) + nlog_2n
            \end{equation*}

            \item Simplify
            \begin{equation*}
                n T(1) + nlog_2n = O(nlog_2n)
            \end{equation*}

        \end{enumerate}

    \end{itemize}

\end{itemize}

\end{document}