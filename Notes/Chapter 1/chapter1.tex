\documentclass{article}

\usepackage{amsmath}
\usepackage{graphicx}
\usepackage{tabularx}
\usepackage{multicol}

% Geometry 
\usepackage{geometry}
\geometry{letterpaper, left=15mm, top=20mm, right=15mm, bottom=20mm}

% Fancy Header
\usepackage{fancyhdr}
\renewcommand{\footrulewidth}{0.4pt}
\pagestyle{fancy}
\fancyhf{}
\chead{CSC 360 - Analysis of Algorithms}
\lfoot{CALU Fall 2021}
\rfoot{RDK}

% Add vertical spacing to tables
\renewcommand{\arraystretch}{1.4}

% Macros
\newcommand{\definition}[1]{\underline{\textbf{#1}}}

\newenvironment{rcases}
  {\left.\begin{aligned}}
  {\end{aligned}\right\rbrace}

% Begin Document
\begin{document}

Introduction to Algorithms - 3rd Edition - Thomas Cormen et.al.

\section*{Chapter 1}

\begin{itemize}

    \item An \definition{algorithm} is any well-defined computational procedure that takes some value, or set of values, as \definition{input} and produces some value, or set of values, as \definition{output}.

    \item An algorithm is a tool for solving a well-specified \definition{computational problem}.
    \begin{itemize}
        \item The statement of the problem gives in general terms the desired input/output relationship.
        \item The algorithm describes a specific computational procedure for achieving that relationship.
    \end{itemize}

    \item \definition{The Sorting Problem}
    \begin{description}
        \item Input: A sequence of $n$ numbers $\langle a_1, a_2, \ldots, a_n \rangle$.
        \item Output: A permutation (reordering) $\langle a'_1, a'_2, \ldots, a'_n \rangle$ of the input sequence such that $a'_1 \leq a'_2 \leq \cdots \leq a'_n$.
    \end{description}

    \item A given input sequence is called an \definition{instance of a problem}.

    \item An algorithm is said to be \definition{correct} if, for every input instance, it halts with the correct output; an algorithm \definition{solves} the given computational problem.

    \item Incorrect algorithms can be useful, if the error rate is controlled

    \item What kinds of problems are solved by algorithms?
    \begin{itemize}
        \item Human Genome Project (determining sequences, storage, data analysis)
        \item The Internet (finding efficient routes, search engines)
        \item Ecommerce
        \item Manufacturing (allocating scarce resources in most efficient way)
    \end{itemize}

\end{itemize}

\end{document}