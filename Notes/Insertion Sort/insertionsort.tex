\documentclass[12pt]{article}

\usepackage{amsmath}
\usepackage{graphicx}
\usepackage{tabularx}
\usepackage{multicol}
\usepackage{algpseudocode}
\usepackage{algorithm}

% Geometry 
\usepackage{geometry}
\geometry{letterpaper, left=15mm, top=20mm, right=15mm, bottom=20mm}

% Fancy Header
\usepackage{fancyhdr}
\renewcommand{\footrulewidth}{0.4pt}
\pagestyle{fancy}
\fancyhf{}
\chead{CSC 360 - Analysis of Algorithms}
\lfoot{CALU Fall 2021}
\rfoot{RDK}

% Add vertical spacing to tables
\renewcommand{\arraystretch}{1.4}

% Macros
\newcommand{\definition}[1]{\underline{\textbf{#1}}}

\newenvironment{rcases}
  {\left.\begin{aligned}}
  {\end{aligned}\right\rbrace}

% Begin Document
\begin{document}

\begin{center}
    \section*{Insertion Sort}
\end{center}

\begin{algorithm}
    \caption{Insertion-Sort($A$)}
    \begin{algorithmic}
        \For{$j = 2$ to $A.Length$}
            \State $key = A[j]$ 
            \\ \Comment Insert $A[j]$ into the sorted sequence $A[1 \ldots j-1]$.
            \State $i = j-1$
            \While{$i > 0$ and $A[i] > key$}
                \State $A[i+1] = A[i]$
                \State $i = i - 1$
            \EndWhile
            \State $A[i+1] = key$
        \EndFor
    \end{algorithmic}
\end{algorithm}

\begin{description}
    \item[Loop Invariant:] At the start of each iteration of the \textbf{for} loop, the subarray $A[1 \ldots j-1 ]$ consists of the elements original in $A[1 \ldots j-1]$, but in sorted order.

    \item[Initialization:] We start by showing that the loop invariant holds before the first loop iteration, when $j=2$. The subarray $A[1 \ldots j-1]$, therefore, consists of just the single element $A[1]$, which is in fact the original element in $A[1]$. Moreover, this subarray is sorted (trivially), which shows that the loop invariant holds prior to the first iteration of the loop.

    \item[Maintenance:] Informally, the body of the \textbf{for} loop works by moving $A[j-1], A[j-2], A[j-3], \ldots$ by one position to the right until it ifnds the proper position for $A[j]$. The subarray $A[1 ldots j-1]$ then consists of the elements originally in $A[1 \ldots j]$, but in sorted order. Incrementing $j$ for the next iteration of the \textbf{for} loop then preserves the loop iteration.
    \footnote{A more formal treatment of the Maintenance property would require us to state and show a loop invariant for the \textbf{while} loop.}

    \item[Termination:] The condition causing the \textbf{for} loop to terminate is that $j > A.length = n$. Because each loop iteration increases $j$ by $1$, we must have $j = n + 1$ at that time. Substituting $n + 1$ for $j$ in the wording of loop invariant, we hae that the subarray $A[1 \ldots n]$ consists of the elements originally in $A[1 \ldots n]$,
    but in sorted order. Observing that the subarray $A[1 \ldots n]$ is the entire array, we conclude that the entire array is sorted.
\end{description}

\end{document}