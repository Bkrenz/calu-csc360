\documentclass[12pt]{article}

\usepackage{amsmath}
\usepackage{graphicx}
\usepackage{tabularx}
\usepackage{multicol}
\usepackage{algpseudocode}
\usepackage{algorithm}

% Geometry 
\usepackage{geometry}
\geometry{letterpaper, left=15mm, top=20mm, right=15mm, bottom=20mm}

% Fancy Header
\usepackage{fancyhdr}
\renewcommand{\footrulewidth}{0.4pt}
\pagestyle{fancy}
\fancyhf{}
\chead{CSC 360 - Analysis of Algorithms}
\lfoot{CALU Fall 2021}
\rfoot{RDK}

% Add vertical spacing to tables
\renewcommand{\arraystretch}{1.4}

% Macros
\newcommand{\definition}[1]{\underline{\textbf{#1}}}

\newenvironment{rcases}
  {\left.\begin{aligned}}
  {\end{aligned}\right\rbrace}

% Begin Document
\begin{document}


\section*{Notes Week 7}

\subsection*{Examples: Recurrences}

\begin{itemize}

    \item Recurrence: $T(n) = 2T(n-1) + 1 \rightarrow O(2^n) \rightarrow$ Fibonacci Numbers

    \item Recurrence: $T(n) = T(n-2) + n \rightarrow O(b^2) \rightarrow$ Fun, made up problem

\end{itemize}

\subsection*{Solving Recurrences by the Master Method}
If the recurrence takes the form: 

\begin{equation*}
    T(n) = aT(\frac{n}{b}) + \Theta(n^k)
\end{equation*}    

where $a \geq 1$ and $b > 1$ then

\begin{center}
    \begin{tabular}{r l}
        $O(n^k)$ & if $a < b^k$ \\
        $O(n^klog_2n)$ & if $a == b^k$ \\
        $O(n^{log_ba})$ & if $ a > b^k$
    \end{tabular}
\end{center}

\subsubsection*{Examples}

\begin{itemize}
    \item Recurrence: $T(n) = 9T(\frac{n}{3}) + n$ \\
    $a = 9, b = 3, k = 1 \rightarrow$ Since $a > b^k$, $O(n^{log_39}) = O(n^2)$

    \item Recurrence: $2T(\frac{n}{2}) + n$

    \item Recurrence: $T(\frac{n}{2}) + 1$

\end{itemize}

\subsection*{Sorting in Linear Time}

\begin{itemize}

    \item Radix sort, when the number of elements far outweighs the digits
    \item Counting sort, when number range is small and only need to count the occurrences

\end{itemize}

\subsubsection*{Good test question:}
When is Radix sort better than merge sort?
\begin{itemize}
    \item Radix Sort $ = O(n \times p)$
    \item Merge Sort $ = O(nlog_2n)$
    \item Radix is faster when $p < log_2n$
\end{itemize}

\subsection*{Graphs}

\begin{itemize}

    \item \definition{Graph}: a set that contains a set of vertices and a set of edges, notated $G = (V,E)$
    \begin{itemize}
        \item $V = $ set of vertices
        \item $E = $ set of edges
    \end{itemize}

    \item \definition{Degree of a Vertex}: number of edges touching a vertex
    \begin{itemize}
        \item \definition{In-Degree}: number of edges traveling into a vertex
        \item \definition{Out-Degree}: numbers of edges traveling out of a vertex
        \item Degrees = In-Degree + Out-Degree
    \end{itemize}

    \item A graph is \definition{Connected} if you can travel from any vertex to any other vertex.

    \item A graph is \definition{Disjoint} if it is not Connected.

    \item Graphs are equivalent if their vertex and edge sets are equivalent

    \item A \definition{Complete Graph} is a graph where all possible edges exist.

    \item A \definition{Planar Graph} is a graph that you can draw in two dimensions such that no edges cross.

    \item Edges and paths can have \definition{weight}, which would be application specific

\end{itemize}


\end{document}