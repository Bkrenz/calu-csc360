\documentclass[12pt]{article}

\usepackage{amsmath}
\usepackage{graphicx}
\usepackage{tabularx}
\usepackage{multicol}
\usepackage{algpseudocode}
\usepackage{algorithm}

% Geometry 
\usepackage{geometry}
\geometry{letterpaper, left=15mm, top=20mm, right=15mm, bottom=20mm}

% Fancy Header
\usepackage{fancyhdr}
\renewcommand{\footrulewidth}{0.4pt}
\pagestyle{fancy}
\fancyhf{}
\chead{CSC 360 - Analysis of Algorithms}
\lfoot{CALU Fall 2021}
\rfoot{RDK}

% Add vertical spacing to tables
\renewcommand{\arraystretch}{1.4}

% Macros
\newcommand{\definition}[1]{\underline{\textbf{#1}}}

\newenvironment{rcases}
  {\left.\begin{aligned}}
  {\end{aligned}\right\rbrace}

% Begin Document
\begin{document}

\section*{Notes Week 4}

\begin{itemize}

    \item Exam in Week 5 - Last Names A-M 6:00pm, N-Y 6:45pm
    
    \item Read Chapter 4 for Week 6 - after exam

\end{itemize}

\section*{Quiz 1 Review - Proof by Induction}

Prove $n < 2n$ for all $n \geq 1$.

\begin{enumerate}

    \item Base Case: $n=1$ \\ \\
    \begin{tabular}{l}
        $1 < 2\times(1)$ \\ 
        $1 < 2$
    \end{tabular}

    \item Assume this is true for $1, \ldots, k$.

    \item If true for $k$, then it is true for $k+1$ \\ \\ 
    \begin{tabular}{l l}
        $k < 2k$ & Independent Hypothesis \\
        $k + 1 < 2k + 1$ & Add 1 to both sides \\
        $k + 1 < 2k + 1 + 1$ & As this is an equality, can add $1$ to the larger side and hold true \\
        $k + 1 < 2k + 2$ & Simplify \\
        $k + 1 < 2(k + 1)$ & Factor out the two \\
    \end{tabular}

\end{enumerate}

\pagebreak

\section*{Example - Proof by Induction}

Prove that path of length $n$ has $n-1$ edges.

\begin{enumerate}

    \item Base Case: $n=2$ \\ \\
    \begin{tabular}{l}
        
    \end{tabular}

    \item Assume this is true for $1, \ldots, k$.

    \item If true for $k$, then it is true for $k+1$ \\ \\ 
    \begin{tabular}{l l}
        
    \end{tabular}

\end{enumerate}

\pagebreak

\section*{Proof By Contradiction}

\begin{itemize}

    \item Prove something by contradicting the negated hypothesis
    
    \item ``Not everyone that is exactly six feet tall has red hair.''

    \item The Steps:
    \begin{enumerate}

        \item Assume that the hypothesis is false

        \item Show the assumption leads to a contradiction

        \item Recognize that assuming step 1 is false leads to a contradiction, implying the hypothesis is true

    \end{enumerate}


\end{itemize}

\section*{Example - Proof by Contradiction}

Prove if $n$ is an integer and $n^2$ is even, then $n$ is even.

Suppose by way of contradiction that: 
\begin{itemize}
    \item $n$ is an integer
    \item $n^2$ is even 
    \item $n$ is odd
    \begin{itemize}
        \item[$\rightarrow$] $n = 2k+1$ for some integer $k$ (the definition of an odd number)
    \end{itemize}
    \item $n^2 = (2k+1)^2 = 4k^2 + 4k + 1$
    \item $2(2k^2+2k) + 1$ is odd
    \item $n^2$ is odd
\end{itemize}

Contradiction: $n^2$ cannot be both even and odd.

\section*{Example - Proof by Contradiction}

Prove: for every $n$, if $n > 2$ and $n$ is prime, then $n$ is odd.

Suppose by way of contradiction that:
\begin{itemize}
    \item $n>2$
    \item $n$ is prime
    \item $n$ is even
    \item $n = 2k$ for some integer $k$ (definition of an even integer) 
    \item $k != 1$
    \item $n = 2(k)$, $k>1$
    \item $n \& (whole number > 1)$ is composite
\end{itemize}

Contradiction: $n$ cannot be both composite and prime


\section*{Example - Proof by Contradiction}

Prove: No integers $a$ \& $b$ exist for which $24a + 12b = 1$.

Suppose by way of contradiction:
\begin{itemize}
    \item $a$ is an integer
    \item $b$ is an integer
    \item $24a + 12b = 1$
    \item Divide by 12, $2a + b = \frac{1}{12}$
\end{itemize}

Contradiction: A whole number multiplied by $2$ summed with another whole number cannnot be a fraction.


\pagebreak

\section*{Notation notes}

\begin{itemize}
    \item $B = O(A) \rightarrow$ B is upper bounded by A
    \item $T(n) = O(f(n))$ if there are positive constants $c$ \& $n_0$ such that $T(n) \leq cf(n)$ when $n \geq n_0$
    \item $A = \Omega(B) \rightarrow$ A is lower bounded by B
    \item $T(n) = \Omega(f(n))$ if there are positive constants $c$ \& $n_0$ such that $T(n) \geq cf(n)$ when $n \geq n_0$
    \item $\omega(n)$ instead of $\Omega(n)$ if not possible to be equal
    \item $T(n) = \mu(f(n))$ if there are positive constants $c$ \& $n_0$ such that $T(n) > cf(n)$ when $n \geq n_0$
    \item Use the lower case versions when bounded but not able to equal. Use upper case when bounded and can be equal, based on definitions above.
    \item $T(n) = \Theta(f(n))$ if and only if $T(n) = O(f(n))$ and $T(n) = \Omega(f(n))$
\end{itemize}

Given $f(n) = 3n^2$ and $g(n) = 5n^2$, give all true statements.
\begin{itemize}
    \item $f(n) = O(g(n))$
    \item $g(g) = \Omega(f(n))$
    \item etc.
\end{itemize}

\end{document}