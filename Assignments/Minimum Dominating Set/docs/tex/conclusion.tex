\documentclass[paper.tex]{subfiles}

\usepackage{tikz}
\usepackage{amsmath}
\usepackage{graphicx}
\usepackage{tabularx}
\usepackage{multicol}
\usepackage{algpseudocode}
\usepackage{algorithm}

% Add vertical spacing to tables
\renewcommand{\arraystretch}{1.4}

% Begin Document
\begin{document}

\section{Conclusion}

This paper has explored two approaches to finding the Minimum Domating Set of a graph, a brute force approach and an approximation approach.
This problem is an $NP-Complete$ problem.
This follows from two facts about the problem.
First, the best known solution is a brute force approach, checking every possible solution.
Second, the solution is verifiable in polynomial time.

The brute force approach checks every possible combination of nodes of the graph to find a minimal covering.
Thus, for a graph with $n$ nodes, the amount of possible combinations is $2^n$, indicating a runtime complexity of $O(2^n)$.
Our experimental data shows that a modest 37-node graph can take nearly 45 hours to run.
This does come with the guarantee of finding the minimal solution.

The approximation approach is much faster however. 
Utilizing a naive greedy approach, a 37-node graph was near instantaneous for a solution.
For all experimental data, the approximate solution ran orders of magnitude faster than the brute force approach.
The implemented approach ran with a time complexity of $O(n^2)$, much better than the brute force approach.
However, this comes with a major trade-off. 
The approximation algorithm rarely finds a minimal solution, with approximate solutions often being double the size of the found minimal solution.

Whether the brute force approach or the approximate approach is the proper choice is domain specific.
Each project will thus need to weigh its factors in determining if an approximate solution is of sufficient quality when applying these techniques.

\end{document}