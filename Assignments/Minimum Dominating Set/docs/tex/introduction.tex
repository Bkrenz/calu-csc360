\documentclass[paper.tex]{subfiles}

\usepackage{tikz}
\usepackage{amsmath}
\usepackage{graphicx}
\usepackage{tabularx}
\usepackage{multicol}
\usepackage{algpseudocode}
\usepackage{algorithm}

% Add vertical spacing to tables
\renewcommand{\arraystretch}{1.4}

% Begin Document
\begin{document}

\section{Introduction}

In this paper, we explore the Minimum Dominating Set of a Graph and how to find it.
The Minimum Domianting Set of a Graph is the Set of all nodes such that every node or one of its neighbors is in the set.

There are two basic algorithms for finding a minimum set.
The first algorithm uses a brute force approach, checking every possible set of vertices.
The second algorithm uses a greedy approach, using the first available vertex with the highest out-degree.
The greedy approach does not guarantee a minimum set.

This paper is divided into the following sections. 
Section 2 contains background information on the Minimum Dominating Set.
Section 3 contains the algorithms used to solve for the Minimum Dominating Set.
Section 4 contains experimental data when running the algorithm.
Section 5 concludes the paper.

\end{document}