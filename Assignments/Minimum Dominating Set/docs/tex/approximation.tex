\documentclass[paper.tex]{subfiles}

\usepackage{tikz}
\usepackage{amsmath}
\usepackage{graphicx}
\usepackage{tabularx}
\usepackage{multicol}
\usepackage{algpseudocode}
\usepackage{algorithm}

% Add vertical spacing to tables
\renewcommand{\arraystretch}{1.4}

% Begin Document
\begin{document}

\section{Approximation Approach}

As the Minimum Domating Set requires a brute force approach, sufficiently large graphs can have enormous run times.
To counter these problems, approximation algorithms are developed.
These are designed to find an answer that is often considered good enough for the task at hand, while having a much lower complexity.

We explored one such approach.
This approach uses a greedy technique for finding a dominating set.
While the set is not covered, it selects the node with the highest out-degree of nodes that are not yet dominated.
This node gets added to the current list of dominating nodes, and repeats until the entire graph is covered.
This greatly improves our runtime complexity to $O(n^2)$.

To put this concept into pseudocode:

\begin{algorithm}[H]

    \caption{Minimum Dominating Set approximation algorithm}

    \begin{algorithmic}[1]
        \Procedure{Approximate Dominating Set}{}
            \While{the graph is not dominated}
                \For{each node not yet domianted}
                    \State Find its out-degree of nodes not yet dominated
                \EndFor
                \State Get the node with the highest out-degree
                \State Set all of its neighbors to be dominated
            \EndWhile
        \EndProcedure
    \end{algorithmic}

\end{algorithm}

As evident, this algorithm is much simpler.
However, there is no guarantee that this algorithm will produce a minimal set, or even a set that is close to minimal.
Depending on the task at hand, this approximate solution using a greedy technique may be preferred to a guaranteed minimal set.

We show some results of our experimental data in the next section.

\end{document}